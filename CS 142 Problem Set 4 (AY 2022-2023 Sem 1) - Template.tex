\documentclass[addpoints,12pt,answers]{exam}
%\documentclass[addpoints,12pt]{exam}

\usepackage{fullpage,graphicx}
\usepackage{amsmath,amsfonts,amsthm,amssymb,multirow, url}
\usepackage{tikz}
\usepackage{wrapfig}
\usepackage{array}
\usepackage{algorithmic}
\usepackage{enumitem}
\usepackage[]{clrscode3e}

\pagestyle{headandfoot}
\firstpageheadrule
\runningheadrule
\firstpagefootrule
\runningfootrule
\firstpageheader{CMSC 142}{Problem Set 4}{AY 2022-2023, 1st Sem.}
\runningheader{CMSC 142}{Problem Set 4}{AY 2022-2023, 1st Sem.}
\firstpagefooter{Page \thepage of \numpages}{}
{
	UP Manila
}
\runningfooter{Page \thepage of \numpages}{}
{
	UP Manila
}

\begin{document}
	
\noindent
\\
\textbf{Name:} [YOUR NAME HERE] %TODO: put your name here
\\

\begin{questions}
\question
For this problem, use the red-black tree visualization of David Galles\\ 
(https://www.cs.usfca.edu/~galles/visualization/RedBlack.html)
\begin{parts}
	\part[5]
	Let's consider an insertion pattern that is the worst case of insertion for a binary search tree (e.g. keys are inserted in sorted order)

	Starting from an empty red-black tree, insert the values $1, 2, \ldots, 22$ in order. At each stage, record the maximum depth in the tree.
	Provide a list of these maximum depths, in order (i.e., $0, 1, 1, 2, \ldots$ ).

	You can use the ``Animation Speed'' slider at the bottom of the tool to speed up the animation. You may need to use ``Change Canvas Size''
	to be able to see the entire tree.
	\begin{solution}
		% TODO: Place your solution here.
	\end{solution}

	\part[5]
	What do you notice about the left side of tree after all 22 keys have been inserted that prevents the root-to-rightmost-node path from 
	getting too large?
	\begin{solution}
		% TODO: Place your solution here.
	\end{solution}

	\part[5]
	Start with a new tree and them try a series of insertions and deletions. Your goal is to get a tree that has at least one root-to-leaf path
	of 3 nodes that has no branching. That is, there is some path from the root to some node $a$ to some other node $b$, such that $b$ is the 
	only child of $a$, and $b$ itself has no children. Take a screenshot of your final tree and put it as the answer here. 
	No explanation is needed.
	\begin{solution}
		% TODO: Place your solution here.
	\end{solution}

	\part[5]
	Try to repeat the previous taks to get a non-branching root-to-leaf path of 4 nodes rather than 3. You will find that this cannot be done.
	Using the rules of a red-black tree, give an explanation about why this is impossible. Your explanation only needs to handle the 4-node case, 
	but notice that it generalizes to longer lengths and helps explain why the tree cannot become too imbalanced.
	\begin{solution}
		% TODO: Place your solution here.
	\end{solution}
\end{parts}

\question[10]
Recall the AVL tree that was introduced in CMSC 123. Prove that an AVL tree with $n$ nodes has height $O(\log n)$. 
(\textit{Hint}: Prove that an AVL tree of height $h$ has at least $F_h$ nodes, where $F_h$ is the $h$-th Fibonacci number.)
\begin{solution}
	% TODO: Place your solution here.
\end{solution}

\question[30]
\textbf{CODING}
Nearly all modern editors include a spellcheck feature which identifies incorrect words and suggests possible words which the 
user might have meant to type. While modern spellcheck algorithms typically involve machine learning, we can create a simple 
spellchecker by suggesting words which are an \textbf{edit distance} of 1 from the incorrect word.

We say that a word is an edit distance of $m$ from another words if we can transform one of the words into the other by using
a total of $m$ operations. Possible operations are deleting a character, insertin a character, replacing a character, and 
switching two adjacent characters. As $m$ increases, finding words which are an edit distance of $m$ apart becomes 
exponentially harder. However, it turns out, around 80\% of spelling errors are an edit distance of just 1 from the correct word!
Thus, we will be able to create a decent spellchecker by simply suggesting words which are an edit distance of 1 from the incorrect 
word.

\textbf{Task}\\
Using the starter code, implement a simple spellchecker that takes in an input word and returns a list of valid words which are an 
edit distance of 0 or 1 from the input word.
\begin{itemize}
	\item You can assume that the input word will consist only of lowercase alphabetic characters. This means you do not need to consider
	uppercase characters, spaces, punctuation, numbers, etc.

	\item The list of valid words are provided in the starter code. You should not return any words which are not in the list of valid words.
	
	\item You do not need to consider whether the input word is spelled correctly. Your function should still return words which are within 
	an edit distance of 0 or 1 of the input word regardless.

	\item A template for a ternary search tree data structure (an extension of the binary search tree) is provided. You should fill in the 
	functions \texttt{insert} and \texttt{contains} within this data structure. Your spellchecker should make use of this ternary search tree
	data structure to store the list of valid words and to check whether a string is among the stored valid words. 

	\item Both your \texttt{insert} and \texttt{contains} function should have worst-case running time of $O(n + k)$, where $n$ is the number 
	of words in the ternary search tree and $k$ is the number of characters in the word being inserted or searched for.

	\item You do not need to implement your own edit distance function. It's already provided for you within the function \texttt{getNearbyStrings}.
	The function \texttt{getNearbyStrings} takes in an input word and returns a list of strings which are an edit distance of at most 1 away from 
	the input word. It is your task to use the ternary search tree data structure to check whether each of the strings in the list is among the 
	stored valid words.
\end{itemize}

\textbf{Ternary Search Trees}\\
In the lecture, we discussed how we can store and search for values efficiently using a binary search tree. In a binary search tree, each node has
(up to) two children, one on the left and one on the right. The left child has a smaller valud that its parent node and the right child has a larger 
value that its parent node. Binary search trees are great for storing and searching data efficiently; however, if we want to store a huge amount of data,
binary search tree might become quite large and take up a lot of memory. What improvements can we make?

If we want to store a lot of dictionary words, the sheer amout of words and the number of bits required to store each character of each words, 
a binary search tree can be quite memory-inefficient. However, in the set of dictionary words, there is a lot of \textit{repeated information}.
Many words share common prefixes but in a binary search tree we store all of these words separately and thus have to store the full bit-length of 
each of these words. What if we only had to store each prefix once?

This is where ternary search trees come in. In a ternary search tree, each node has \textit{three} child nodes instead of two: a left child, 
a middle child, and a right child. The left child contains a character value less that its parent node, and the right child contains a character value
that is greater than its parent node. The middle child, on the other hand, contains the next characters of a stored word.

How do we insert and search for words? You can visualize how they work with this visualization: https://www.cs.usfca.edu/~galles/visualization/TST.html.
Try using the ``Insert'' function to insert different words in a ternary search tree, including words with and without common prefixes and words with different
starting letters, and try using ``Find'' function to search for words that may or may not be in your ternary search tree. One helpful set of words to try is
[``she'',  ``sells'', ``sea'', ``shells'', ``by'', ``the'', ``sea'', ``shore'']. As you go through the visualization, read through the following step-by-step
descriptions.

\textbf{Insert}
When we insert a word in a ternary search tree, we insert one characters at a time. We first check the middle child node. If none exists, we insert the 
character, traverse down that new node, and move on to the next characters in our word. If a middle child node exists and its value is equal to our character
then we traverse down the middle child node and move on to the next characters in our word.

If a middel child exists but its value is not equal to our characters, then we either traverse down the left child node if our characters has value less than 
the middle child node or traverse down the right child node if our character has value greater than the middel child node. If no such left or right child node exists,
we create it and insert the characters there, and then store the remaining characters in our word in successive middle child nodes.

Additionally, upon inserting the last character of our word, we mark the node containing the last character as an end node. This is important for searching.

\textbf{Search}
When searching for a word in a ternary search tree, we compare one character at a time. We first check the middle child node. If the middle child's value is equal to our character, 
then we traverse down it and move one to the next character in our word. If the middle child's value is less than our character, then we traverse down the left child node. 
If it is equal, we move on to the next character in our word; otherwise, we traverse down the left or right child of that node and repeat. If the middle child's value is greater than 
our character, then we traverse down the right child node. Similarly, if it is equal, we move on to the next character in our word; otherwise, we traverse down the left or right child 
of that node and repeat.

If any of the child nodes we check do not exist, then we return that the word does not exist in the tree. Or if we reach the last character of our word and its node is not marked as 
an end node, then we also return that the word does not exist in the tree. If we reach the last character of our word and its node is marked as an end node, only then do we return 
that the word exists in tree.
\end{questions}
\end{document}